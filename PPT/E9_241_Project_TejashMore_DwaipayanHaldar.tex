\documentclass{beamer}
\usepackage{graphicx} % Package for images
\usepackage{caption} % Package for numbering figures
\usepackage{subcaption}
\setbeamertemplate{caption}[numbered] % Enable numbering for figures
\captionsetup{font=tiny} % Reduce caption size
\usepackage{tikz}
\usepackage{pgfplots}
\usepackage{multicol}
\usepackage{amsmath}
\usepackage{bbm}
\usepackage{dsfont}
\usetikzlibrary{trees, positioning}
% Theme choice
\usetheme{Frankfurt} % Changed theme to Frankfurt
\pgfplotsset{compat=1.18}
\usepackage{helvet} \renewcommand{\familydefault}{\sfdefault}
\setbeamertemplate{headline}{\leavevmode\hbox{}} 

% Title and author details
\title{Wavelet Based Image Denoising}
\author{Tejash More, Dwaipayan Haldar \protect\linebreak M Tech, Signal Processing \protect\linebreak Indian Institute of Science}
\date{\today}

\begin{document}

\begin{frame}
\maketitle
\end{frame}

% ---------- FRAME 1 ----------
\begin{frame}{Introduction}
\begin{itemize}
    \item Spatial denoising smooths pixels directly, so it often blurs edges and fine textures, especially at higher noise levels. It cannot distinguish noise from real high-frequency details.
    \item While wavelet based denoising, noise is spread mostly as low coefficients values in the wavelet domain. Applying the threshold removes the low coefficient noise while preserving the high coefficient edges.
    \item The basic workflow of our denoising algorithm works as follows:
    \begin{figure}[H]
    \centering
    \includegraphics[scale = .27]{Block-diagram-of-Image-denoising-using-Wavelet-Transform.png}
    \caption{Flowchart of the Overall Algorithm}
    \label{fig:flowchart}
    \end{figure}
\end{itemize}



\end{frame}

% ---------- FRAME 2 ----------
\begin{frame}{The Wavelet Transform}
\begin{itemize}
    \item The Discrete Wavelet Transform (DWT) decomposes an image into multiple frequency subbands (LL, HL, LH, HH), capturing coarse structure and fine details at different scales.
    \item  It provides a sparse representation where important features concentrate in a few large coefficients.
    \item Noise mostly appears in the high-frequency bands, making it easy to remove using thresholding.
\end{itemize}
\begin{figure}[h]
    \centering
    \begin{minipage}{0.32\textwidth}
        \centering
        \includegraphics[width=\linewidth]{../outputImage/cameraman_wavelet_haar_lvl_1.png} 
        % \caption{(a)}
        % \label{fig:2a}
    \end{minipage}
    \begin{minipage}{0.32\textwidth}
        \centering
        \includegraphics[width=\linewidth]{../outputImage/lena_wavelet_haar_lvl_1.png}
        % \caption{(b)}
        % \label{fig:2b}
    \end{minipage}
    \caption{LL,HL,LH,HH representation for cameraman and lena with Haar Wavelet (1-Level)}
\end{figure}
\end{frame}

% ---------- FRAME 3 ----------
\begin{frame}{Thresholding / Shrinking Algorithms}
\textbf{VisuShrink}
\begin{itemize}
    \item Simplest of all. Oversmooths. Uses universal threshold: $T = \sigma \sqrt{2 \log n}$, where $\sigma$ is estimated by the robust median estimator $\hat\sigma = \frac{ Median(|Y_{ij}|)}{0.6745}, Y_{ij} \in  \text{subband HH1}$
\end{itemize}

\textbf{SUREShrink}
\begin{itemize}
    \item Uses Stein's Unbiased Risk Estimate(SURE) for estimating bayesian risk. Picks the threshold that minimizes $SURE(t;\mathbf{x})$. It is given by:
\[
T_S = argmin_{t\geq0} SURE(t;\mathbf{x}) = d - 2\cdot\sum_{i=1}^{d}\mathds{1}_{|x_i|\leq t} + \sum_{i=1}^{d} min(x_i^2, y^2)
\]
\end{itemize}

\textbf{BayesShrink}
\begin{itemize}
    \item Minimizes the Bayesian risk using a GGD prior. 
\[
T_B = \frac{\sigma^2}{\sigma_X}; \hat\sigma_X = \sqrt{max(\hat\sigma_Y^2-\hat\sigma^2,0)}; \hat\sigma_Y^2 = \frac{1}{n^2}\sum_{i,j=1}^{n} Y_{ij}^2
\] 
\end{itemize}
\end{frame}

% ---------- FRAME 5 ----------
\begin{frame}{Cameraman: Complete Workflow}
The complete workflow is given cameraman image. Level-1 wavelet decompostion is shown for easy visualization. Experiments are carried on Level-4. Here Haar filter is used. Denoising is done using BayesShrink.

\begin{figure}[h]
    \centering
    \begin{minipage}{0.20\textwidth}
        \centering
        \includegraphics[width=\linewidth]{../outputImage/cameraman_gaussian_noise.png} 
        \subcaption{}
        % \label{fig:2a}
    \end{minipage}
    $\stackrel{\text{\TINY DWT(L-1)}}\rightarrow$
    \begin{minipage}{0.20\textwidth}
        \centering
        \includegraphics[width=\linewidth]{../outputImage/cameraman_wavelet_haar_lvl_1.png}
        \subcaption{}
        % \label{fig:2b}
    \end{minipage}
    $\underset{\text{\TINY Threshold}}{\overset{\text{\TINY Bayesshrink}}{\rightarrow}}$
    \begin{minipage}{0.20\textwidth}
    \centering
    \includegraphics[width=\linewidth]{../outputImage/cameraman_wavelet_haar_lvl_1_bayes_denoised.png}
        \subcaption{}
        % \label{fig:2b}
    \end{minipage}
    $\stackrel{\text{\TINY IDWT(L-1)}}\rightarrow$
    \begin{minipage}{0.20\textwidth}
    \centering
    \includegraphics[width=\linewidth]{../outputImage/cameraman_denoised_haar_level_1_bayes.png}
        \subcaption{}
        % \label{fig:2b}
    \end{minipage}
    \caption{Denoising Using wavelet Transform Flowchart: (a) Denoised Image, (b) DWT with Haar Wavelet Level-1, (c) Thresholded Wavelet coefficients using BayesShrink threshold, (d) Denoised Image}
\end{figure}

\tiny **Noise distributions used for reporting are: Gaussian Noise($\mu = 0, \sigma= 0.25$), Uniform Noise(Range = [-1,1]) and Salt and Pepper Noise(Amount = 0.05).

\end{frame}

% ---------- FRAME 6 ----------
\begin{frame}{Cameraman: Comparison with Haar Wavelet}
\begin{figure}[h]
    \centering
    \begin{minipage}{0.22\textwidth}
        \centering
        \includegraphics[width=\linewidth]{../outputImage/cameraman_gaussian_noise.png} 
        \subcaption{SSIM: 27.99, PSNR: 20.57}
        % \label{fig:2a}
    \end{minipage}
    \begin{minipage}{0.22\textwidth}
        \centering
        \includegraphics[width=\linewidth]{../outputImage/cameraman_gaussian_haar_visu.png}
        \subcaption{SSIM: 70.96, PSNR: 23.97}
        % \label{fig:2b}
    \end{minipage}
    \begin{minipage}{0.22\textwidth}
        \centering
        \includegraphics[width=\linewidth]{../outputImage/cameraman_gaussian_haar_sure.png}
        \subcaption{SSIM: 71.26, PSNR: 27.97}
        % \label{fig:2b}
    \end{minipage}
    \begin{minipage}{0.22\textwidth}
        \centering
        \includegraphics[width=\linewidth]{../outputImage/cameraman_gaussian_haar_bayes.png}
        \subcaption{SSIM: 68.62, PSNR: 27.75}
        % \label{fig:2b}
    \end{minipage}

    \centering
    \begin{minipage}{0.22\textwidth}
        \centering
        \includegraphics[width=\linewidth]{../outputImage/cameraman_uniform_noise.png} 
        \subcaption{SSIM: 43.78, PSNR: 25.02}
        % \label{fig:2a}
    \end{minipage}
    \begin{minipage}{0.22\textwidth}
        \centering
        \includegraphics[width=\linewidth]{../outputImage/cameraman_uniform_haar_visu.png}
        \subcaption{SSIM: 75.87, PSNR: 25.72}
        % \label{fig:2b}
    \end{minipage}
    \begin{minipage}{0.22\textwidth}
        \centering
        \includegraphics[width=\linewidth]{../outputImage/cameraman_uniform_haar_sure.png}
        \subcaption{SSIM: 84.09, PSNR: 30.98}
        % \label{fig:2b}
    \end{minipage}
    \begin{minipage}{0.22\textwidth}
        \centering
        \includegraphics[width=\linewidth]{../outputImage/cameraman_uniform_haar_bayes.png}
        \subcaption{SSIM: 79.21, PSNR: 30.69}
        % \label{fig:2b}
    \end{minipage}
    \caption{Image Comparison: (a) Gaussian Noisy Image, (b) Denoised Image (Gaussian, VisuShrink), (c) Denoised Image (Gaussian, SureShrink), (d) Denoised Image (Gaussian, BayesShrink), (e) Uniform Noisy Image, (f) Denoised Image (Uniform, VisuShrink), (g) Denoised Image (Uniform, SureShrink), (h) Denoised Image (Uniform, BayesShrink)}
\end{figure}


\end{frame}

% ---------- FRAME 7 ----------
\begin{frame}{Cameraman: Comparison with Db4 Wavelet}
\begin{figure}[h]
    \centering
    \begin{minipage}{0.22\textwidth}
        \centering
        \includegraphics[width=\linewidth]{../outputImage/cameraman_gaussian_noise.png} 
        \subcaption{SSIM: 27.99, PSNR: 20.57}
        % \label{fig:2a}
    \end{minipage}
    \begin{minipage}{0.22\textwidth}
        \centering
        \includegraphics[width=\linewidth]{../outputImage/cameraman_gaussian_db4_visu.png}
        \subcaption{SSIM: 71.88, PSNR: 24.91}
        % \label{fig:2b}
    \end{minipage}
    \begin{minipage}{0.22\textwidth}
    \centering
    \includegraphics[width=\linewidth]{../outputImage/cameraman_gaussian_db4_sure.png}
        \subcaption{SSIM: 73.40, PSNR: 29.17}
        % \label{fig:2b}
    \end{minipage}
    \begin{minipage}{0.22\textwidth}
        \centering
        \includegraphics[width=\linewidth]{../outputImage/cameraman_gaussian_db4_bayes.png}
        \subcaption{SSIM: 73.00, PSNR: 29.28}
        % \label{fig:2b}
    \end{minipage}

    \centering
    \begin{minipage}{0.22\textwidth}
        \centering
        \includegraphics[width=\linewidth]{../outputImage/cameraman_uniform_noise.png} 
        \subcaption{SSIM: 43.78, PSNR: 25.02}
        % \label{fig:2a}
    \end{minipage}
    \begin{minipage}{0.22\textwidth}
        \centering
        \includegraphics[width=\linewidth]{../outputImage/cameraman_uniform_db4_visu.png}
        \subcaption{SSIM: 76.87, PSNR: 26.89}
        % \label{fig:2b}
    \end{minipage}
    \begin{minipage}{0.22\textwidth}
    \centering
    \includegraphics[width=\linewidth]{../outputImage/cameraman_uniform_db4_sure.png}
        \subcaption{SSIM: 85.39, PSNR: 32.41}
        % \label{fig:2b}
    \end{minipage}
    \begin{minipage}{0.22\textwidth}
        \centering
        \includegraphics[width=\linewidth]{../outputImage/cameraman_uniform_db4_bayes.png}
        \subcaption{SSIM: 81.69, PSNR: 32.09}
        % \label{fig:2b}
    \end{minipage}
    \caption{Image Comparison: (a) Gaussian Noisy Image, (b) Denoised Image (Gaussian, VisuShrink), (c) Denoised Image (Gaussian, SureShrink), (d) Denoised Image (Gaussian, BayesShrink), (e) Uniform Noisy Image, (f) Denoised Image (Uniform, VisuShrink), (g) Denoised Image (Uniform, SureShrink), (h) Denoised Image (Uniform, BayesShrink)}
\end{figure}
\end{frame}

% ---------- FRAME 8 ----------
\begin{frame}{Cameraman: Comparison with Spatial Filter Image}
\begin{figure}[h]
    \centering
    \begin{minipage}{0.25\textwidth}
        \centering
        \includegraphics[width=\linewidth]{../outputImage/cameraman_gaussian_db4_visu.png}
        \subcaption{SSIM: 71.88, PSNR: 24.91}
        % \label{fig:2a}
    \end{minipage}
    \begin{minipage}{0.25\textwidth}
        \centering
        \includegraphics[width=\linewidth]{../outputImage/cameraman_gaussian_db4_sure.png}
        \subcaption{SSIM: 73.40, PSNR: 29.17}
        % \label{fig:2b}
    \end{minipage}
    \begin{minipage}{0.25\textwidth}
        \centering
        \includegraphics[width=\linewidth]{../outputImage/cameraman_gaussian_db4_bayes.png}
        \subcaption{SSIM: 73.00, PSNR: 29.28}
        % \label{fig:2b}
    \end{minipage}
    \begin{minipage}{0.25\textwidth}
        \centering
        \includegraphics[width=\linewidth]{../outputImage/cameraman_denoised_spatial.png} 
        \subcaption{SSIM: 75.78, PSNR: 27.54}
        % \label{fig:2b}
    \end{minipage}

    \caption{Image Comparison: (a) Denoised Image (Gaussian, VisuShrink), (b) Denoised Image (Gaussian, SureShrink), (c) Denoised Image (Gaussian, BayesShrink), (d) Denoised Image (Gaussian Spatial Filter)}
\end{figure}
\end{frame}

% -----------------------------

% ---------- FRAME 9 ----------
\begin{frame}{Quantitative Comparison (Standard Image Dataset)}
\small Comparison of SSIM plots of 12 standard test images with 2 Wavelet based thresholding methods(Db4) and spatial domain filtering(Gaussian Filter, Median Filter) is calculated for different Gaussian Noise, Salt and Pepper Noise and Uniform Noise.\\
\phantom{123}
\begin{figure}[h]
    \centering
    \begin{minipage}{0.4\textwidth}
        \centering
        \includegraphics[width=\linewidth]{../outputImage/comparison_plot_gaussian.png} 
        \subcaption{}
        % \label{fig:2a}
    \end{minipage}

    \begin{minipage}{0.4\textwidth}
        \centering
        \includegraphics[width=\linewidth]{../outputImage/comparison_plot_uniform.png}
        \subcaption{}
        % \label{fig:2b}
    \end{minipage}
    \begin{minipage}{0.4\textwidth}
    \centering
    \includegraphics[width=\linewidth]{../outputImage/comparison_plot_sp.png}
        \subcaption{}
        % \label{fig:2b}
    \end{minipage}

    \caption{(a) Gaussian, (b)Uniform, (c) Salt and Pepper Noise}
\end{figure}

\end{frame}

% ---------- FRAME 10 (fixed) ----------
\begin{frame}{Discussion}
\begin{itemize}
    \item \textbf{Wavelet:} One observation while operating with different mother wavelets is that \textit{Db4} performs better than \textit{Haar} in almost all the cases. That is expected since \textit{Db4} is smoother and more correlated to natural images than \textit{Haar} wavelet. 
    \item \textbf{Noise:} While wavelet domain denoising performs pretty well for Gaussian and Uniform noise, it fails for Salt and Pepper Noise and lags much behind Median filters. This happens due to the prior assumption of Gaussian Noise in all the thresholding methods.
    \item \textbf{Thresholding:} SUREShrink and BayesShrink almost goes parallel to each other. That is also evident from the \cite{b5} where the author compares BayesShrink and SUREShrink, that comes out to be within 8\% of each other. VisuShrink is universal thresholding it oversmooths images in most cases, giving below average results.
\end{itemize}
\end{frame}

% ---------- FRAME 11 ----------
\begin{frame}{References}
\begin{thebibliography}{00}

\bibitem{b1} S. Mallat, \emph{A Wavelet Tour of Signal Processing}. San Diego, CA, USA: Academic Press, 1999.

\bibitem{b2} D. L. Donoho, ``De-noising by soft-thresholding,'' \emph{IEEE Trans. Inf. Theory}, vol.~41, no.~3, pp.~613--627, 1995.

\bibitem{b3} D. L. Donoho and I. M. Johnstone, ``Adapting to unknown smoothness via wavelet shrinkage,'' \emph{J. Amer. Stat. Assoc.}, vol.~90, no.~432, pp.~1200--1224, 1995.

\bibitem{b4} D. L. Donoho and I. M. Johnstone, ``Threshold selection for wavelet shrinkage of noisy data,'' in \emph{Proc. 16th Annu. Int. Conf. IEEE Eng. Med. Biol. Soc.}, vol.~1, 1994, pp.~A24--A25.

\bibitem{b5} S. G. Chang, B. Yu, and M. Vetterli, ``Adaptive wavelet thresholding for image denoising and compression,'' \emph{IEEE Trans. Image Process.}, vol.~9, no.~9, pp.~1532--1546, 2000.

\end{thebibliography}
\end{frame}

%------------------------------


\begin{frame}{}
    \centering
    \Huge{Thank You!}
    \vspace{1cm}
    % \\ \Large{Questions?}
\end{frame}


\end{document}